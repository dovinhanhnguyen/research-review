\documentclass[english]{article}
\usepackage[utf8]{inputenc}
\usepackage[T1]{fontenc}
\usepackage{babel}
\usepackage{amsmath}
\usepackage{graphicx}
\usepackage{fancyhdr}
\pagestyle{fancy}
\fancyhf{}
\renewcommand{\headrulewidth}{0pt}
\setlength{\headheight}{40pt} 

\begin{document}

\title{Looking beyond the Standard Model: \\Neutrino oscillations - history and current status}
\author{}
\date{April 30, 2018}
\maketitle

\thispagestyle{fancy}

\begin{abstract}
In this section
\end{abstract}

\section{History}
	Pauli first proposed the existence of neutrinos ("neutrons") in 1930 \cite{pauliletter1930} when he was looking at the problem of radioactive $\beta$-decay, in which the emitted electrons had a continuous spectrum of energies leading to contradiction with the principle of energy conservation. Pauli suggested that there must be another unseen particle, of spin $1/2$ and mass of the same order of magnitude as the electron mass, emitted along with the electron in order for energy to be conserved.
    
    In 1934, Fermi used Pauli's idea as the basis of his famous theory of $\beta$-decay and generally theory of weak interaction \cite{fermi1934}, coining the name "neutrino" ("little neutral one") in the process. Fermi was then able to calculate the probability of neutrino detection, which came out to be very small and prompted Bethe and Peierls to claim that neutrinos might never be observed \cite{bethepeierls1934}. However, in 1956 Cowan and Reines succeeded in doing just that \cite{cowanreines1956} by their discovery of the antineutrinos in nuclear reactor using the reaction
    \begin{gather}
    	\bar{\nu}+p \rightarrow n+e^{+}
    \end{gather}
    
    abc

\section{Background}

	Bubuuu and Beebeeeeeeeee

\begin{thebibliography}{1}

	\bibitem{pauliletter1930}
    Pauli W., 1930. "Open letter to the group of radioactive people at the Gauverein meeting in T\"{u}bingen". Pauli Archive at CERN.
    \bibitem{fermi1934}
    Fermi E., 1934. "Tentativo Di Una Teoria Dei Raggi $\beta$". \textit{Nuovo Cim.}, 11, 1.
    \bibitem{bethepeierls1934}
    Bethe H., Peierls R., 1934. \textit{Nature}, 133, 532.
    \bibitem{cowanreines1956}
    Cowan C. L. textit{et al.}, 1956. "Detection of the Free Neutrino: a Confirmation". \textit{Science}, 124, 103.
    
\end{thebibliography}



\end{document}