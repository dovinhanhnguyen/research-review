\documentclass[english]{article}
\usepackage[utf8]{inputenc}
\usepackage[T1]{fontenc}
\usepackage{babel}
\usepackage{amsmath}
\usepackage{graphicx}
\usepackage{fancyhdr}
\pagestyle{fancy}
\fancyhf{}
\renewcommand{\headrulewidth}{0pt}
\setlength{\headheight}{40pt} 

\begin{document}

\title{Looking beyond the Standard Model: \\Neutrino oscillations - history and current status}
\author{Candidate Number: 6950X}
\date{Suprevised by Dr Tina Potter}
\maketitle

\thispagestyle{fancy}

\begin{abstract}
In this section
\end{abstract}

\section{Neutrinos and the Standard Model}
	Pauli first proposed the existence of neutrinos ("neutrons") in 1930 \cite{pauliletter1930} when he was looking at the problem of radioactive $\beta$-decay, in which the emitted electrons had a continuous spectrum of energies leading to contradiction with the principle of energy conservation. Pauli suggested that there must be another unseen particle, of spin $1/2$ and mass of the same order of magnitude as the electron mass, emitted along with the electron in order for energy to be conserved.
    
    In 1934, Fermi used Pauli idea as the basis of his famous theory of $\beta$-decay and generally theory of weak interaction \cite{fermi1934}, coining the name "neutrino" ("little neutral one") in the process. Fermi was then able to calculate the probability of neutrino detection, which came out to be so small that it prompted Bethe and Peierls to claim that neutrinos might never be observed \cite{bethepeierls1934}. However, in 1956 Cowan and Reines succeeded in doing just that \cite{cowanreines1956} by their discovery of the antineutrinos in nuclear reactor using the reaction
    \begin{gather}
    	\bar{\nu}+p \rightarrow n+e^{+}
    \end{gather}
    
    After parity was found not to be conserved in the $\beta$-decay and other weak processes \cite{wu1957} by Wu \textit{et al.} in 1957, Salam \cite{salam1956}, Landau \cite{landau1957}, Lee and Yang \cite{leeyang1957} put forward the theory of the two-component neutrino using Weyl previously rejected idea of two-component spinors \cite{weyl1929}. Consider the Dirac equation for the neutrino field with mass $m_{\nu}$
    \begin{gather}
    	i\gamma^{\alpha} \partial_{\alpha} \nu (x) - m_{\nu} \nu (x) = 0
    \end{gather}
    
    For left-handed and right-handed components, $\nu_{L} (x)$ and  $\nu_{R} (x)$, we obtain two coupled equations
    \begin{gather}
    	i\gamma^{\alpha} \partial_{\alpha} \nu_{L} (x) - m_{\nu} \nu_{R} (x) = 0 \\
        i\gamma^{\alpha} \partial_{\alpha} \nu_{R} (x) - m_{\nu} \nu_{L} (x) = 0
    \end{gather}
    
    Salam, Landau, Lee and Yang chose to assume that neutrino mass is zero, which is a reasonable assumption given the data existed at the time. If this is the case, we have two decoupled Weyl equations
    \begin{gather}
    	i\gamma^{\alpha} \partial_{\alpha} \nu_{L,R} (x) = 0
    \end{gather}
    and then the neutrino field can either be $\nu_{L} (x)$ or $\nu_{R} (x)$.
    
    The two-component theory implies the parity violation in $\beta$-decay and other weak processes (in agreement with the experimental results of the Wu \textit{et al.} and other experiments \cite{wu1957} \cite{garwinledermanweinrich1957}), and neutrino (antineutrino) helicity is equal to $-1$ ($+1$) if the field is $\nu_{L} (x)$ and is equal to $+1$ ($-1$) if the field is $\nu_{R} (x)$.
    
    In 1958, the helicity of neutrinos was measured from the chain reaction
    \begin{gather}
    	e^{-} + {}^{152} Eu \rightarrow {}^{152} Sm^{*} + \nu_{e} \\
        {}^{152} Sm^{*} \rightarrow {}^{152} Sm + \gamma
    \end{gather}
    by Goldhaber \textit{et al.} \cite{goldhabergrodzinssunyar1958}. The neutrino helicity was negative in full agreement with the two-component theory of massless neutrino, and it looks like from the two possibilities, $\nu_{L} (x)$ or $\nu_{R} (x)$, nature pick the first one.
    
    The two-component theory was built on the assumption that neutrinos have vanishing mass. This point of view was challenged after Feynman and Gell-Mann \cite{feynmangellmann1958}, Sudarshan and Marshak \cite{sudarshanmarshak1958} proposed their $V - A$ theory in 1958, suggesting that the violation of parity in the weak interaction is not connected with exceptional properties of neutrinos. Nevertheless, the two-component theory of massless neutrino was the simplest theoretical possibility and it still produced predictions that were consistent with the contemporary experiments on weak processes.
    
    In the following few years, the theory of electroweak interactions was formulated under the assumption of massless two-component neutrinos \cite{glashow1961} \cite{goldstonesalamweinberg1962} \cite{weinberg1967}. Together with the theory of the strong interaction \cite{grosswilczek1973} \cite{politzer1973}, the full theory describing all elementary particle interactions is known as the \textit{Standard Model}.

\section{Neutrino Oscillations}
	The problems of the Standard Model picture of neutrinos began with the Homestake experiment headed by Davis \cite{davis1968}. In 1968, Davis attempted to detect the solar neutrinos based upon the reaction
    \begin{gather}
    	\nu_{e} + {}^{37} Cl \rightarrow e^{-} + {}^{37} Ar
    \end{gather}
    A 380 cubic meter tank of a fluid rich in chlorine called perchloroethylene was placed 1,478 meters deep underground in the Homestake Gold Mine in South Dakota, USA to prevent interference from cosmic rays. Helium was bubbled through the fluid periodically to remove the argon that had formed, which were then counted by means of their radioactivity to determine how many neutrinos had been captured.
    
    Although solar neutrinos were successfully detected by Davis, a new problem emerged. The experimental results were consistently very close to one-third of Bahcall's calculations, that is the flux of neutrinos found by the detector was only one-third the amount theoretically predicted by the Standard Solar Model \cite{davis1998} \cite{bahcall2004}. This discrepancy was known as the \textit{Solar Neutrino Problem}. Many physicists believed that the solution lies with a wrong neutrino flux given by the Standard Solar Model or a mistake made by Davis while carrying out the experiment, but not with the Standard Model. However, other subsequent experiments with the same purpose such as Kamiokande and later Super-Kamiokande in Japan e.g.\cite{kamiokande1991} \cite{superk2016}, SAGE in the former Soviet Union e.g.\cite{sage1991}, GALLEX in Italy e.g.\cite{gallex1999}, and SNO in Canada e.g.\cite{sno2001} confirmed the results of Davis.
    
    Similar alarming findings were found in the atmospheric neutrino flux, with several experimental groups observing a deficit in the number of atmospheric neutrinos produced by cosmic rays impacting on the Earth’s atmosphere e.g\cite{hirata1998} \cite{casper1991} \cite{macro1998}. This became the \textit{Atmospheric Neutrino Anomaly}.
    
    By the end of the nineties, it appeared likely that the Standard Model has to be extended to explain these phenomena. The theory required, \textit{neutrino oscillations}, however, had been invented as long ago as 1957 by Pontecorvo \cite{pontecorvo1957}, who suggested that $\nu \Leftrightarrow \bar{\nu}$ transitions can occur in analogy with the $K^{0} \Leftrightarrow \bar{K}^{0}$ oscillation proposed earlier by Gell-Mann and Pais in 1955 \cite{gellmann1955}. Based on this idea, the theory of flavour neutrino mixing was first developed by Maki, Nakagawa, and Sakata in 1962 \cite{mns1962}, in which they assumed that the flavour eigenstates $\nu_{e}$ and $\nu_{\mu}$ ($\nu_{\tau}$ had yet to be discovered at the time) are not mass eigenstates, but are superposition of two "true neutrinos" with different masses
    \begin{gather}
    	\nu_{e} = \nu_{1} cos\delta + \nu_{2} sin\delta \\
        \nu_{\mu} = -\nu_{1} sin\delta + \nu_{2} cos\delta
    \end{gather}
    through some orthogonal transformation characterised by an angle $\delta$. The significance of this theory is that in order for mass to be a valid labeling scheme, it suggests that \textit{neutrinos must have finite mass}. This is different from the assumption of the two-component massless neutrino theory that the Standard Model was previously built upon.
    
    The theory was further elaborated by Pontecorvo in 1967 \cite{pontecorvo1967}. In 1969, a year after the first solar neutrino deficit was observed in the previously mentioned Homestake experiment, Gribov and Pontecorvo followed up by publishing another paper \cite{pontecorvo1969}, in which they described quantitatively the idea of $\nu_{e} \Leftrightarrow \nu_{\mu}$ oscillations and how it might explain the decrease in the number of detectable solar neutrinos at the Earth surface. The first full version of the theory was worked out in several papers in the seventies \cite{fulltheory70s}. However, judging from the number of publications that are still emerging, this subject is still up for debate.
    
    All existing data on neutrino oscillations can be described assuming
3-flavour neutrino mixing in vacuum. In its simplest form it can be expressed as a unitary transformation relating the flavour and mass eigenstates
	\begin{gather}
    	\nu_{lL} (x) = \sum_{j=1}^{3} U_{lj} \nu_{jL} (x)
    \end{gather}
    where
    
\begin{thebibliography}{99}

	\bibitem{pauliletter1930}
    Pauli W., 1930. "Open letter to the group of radioactive people at the Gauverein meeting in T\"{u}bingen". \textit{Pauli Archive at CERN}.
    \bibitem{fermi1934}
    Fermi E., 1934. "Tentativo Di Una Teoria Dei Raggi $\beta$". \textit{Nuovo Cim.}, 11, 1.
    \bibitem{bethepeierls1934}
    Bethe H., Peierls R., 1934. \textit{Nature}, 133, 532.
    \bibitem{cowanreines1956}
    Cowan C. L. \textit{et al.}, 1956. "Detection of the Free Neutrino: a Confirmation". \textit{Science}, 124, 103.
    \bibitem{wu1957}
    Wu C. S. \textit{et al.}, 1957. "Experimental Test of Parity Conservation in Beta Decay". \textit{Phys. Rev.}, 105, 1413.
    \bibitem{salam1956}
    Salam A., 1957. "On Parity Conservation and Neutrino Mass". \textit{Nuovo Cim.}, 5, 299.
    \bibitem{landau1957}
    Landau L., 1957. "On the Conservation Laws for Weak Interactions". \textit{Nucl. Phys.}, 3, 127.
    \bibitem{leeyang1957}
    Lee. T. D., Yang C. N., 1957. "Parity Nonconservation and a Two-Component Theory of the Neutrino". \textit{Phys. Rev.}, 105, 1671.
    \bibitem{weyl1929}
    Weyl H., 1929. "Elektron und Gravitation. I.". \textit{Z. Physik}, 56, 330.
    \bibitem{garwinledermanweinrich1957}
    Garwin R. L., Lederman L. M., Weinrich M., 1957. "Observations of the Failure of Conservation of Parity and Charge Conjugation in Meson Decays: the Magnetic Moment of the Free Muon". \textit{Phys. Rev.}, 105, 1415.
    \bibitem{goldhabergrodzinssunyar1958}
    Goldhaber M., Grodzins L., Sunyar A. W., 1958. "Helicity of Neutrinos". \textit{Phys. Rev.}, 109, 1015.
    \bibitem{feynmangellmann1958}
    Feynman R. P., Gell-Mann M., 1958. "Theory of the Fermi Interaction". \textit{Phys. Rev.}, 109, 193.
    \bibitem{sudarshanmarshak1958}
    Sudarshan E. C. G., Marshak R. E., 1958. "Chirality Invariance and the Universal Fermi Interaction". \textit{Phys. Rev.}, 109, 1860.
    \bibitem{glashow1961}
    Glashow S. L., 1961. "Partial-symmetries of Weak Interactions". \textit{Nucl. Phys.}, 22, 579.
    \bibitem{goldstonesalamweinberg1962}
    Goldstone J., Salam A., Weinberg S., 1962. "Broken Symmetries". \textit{Phys. Rev.}, 127, 965.
    \bibitem{weinberg1967}
    Weinberg S., 1967. "A Model of Leptons". \textit{Phys. Rev. Lett.}, 19, 1264.
    \bibitem{grosswilczek1973}
    Gross D. J., Wilczek F., 1973. "Ultraviolet Behavior of Non-Abelian Gauge Theories". \textit{Phys. Rev. Lett.}, 30, 1343.
    \bibitem{politzer1973}
    Politzer H. D., 1973. "Reliable Pertubative Results for Strong Interactions?". \textit{Phys. Rev. Lett.}, 30, 1346.
    \bibitem{davis1968}
    Davis R., Harmer D. S., Hoffman K. C., 1968. "Search for Neutrinos from the Sun". \textit{Phys. Rev. Lett.}, 20, 1205.
    \bibitem{davis1998}
    Cleveland B. T. \textit{et al.}, 1998. "Measurement of the Solar Electron Neutrino Flux with the Homestake Chlorine Detector". \textit{The Astrophys. Jour.}, 496, 505.
    \bibitem{bahcall2004}
    Bahcall J. N., Pinsonneault M. H., 2004. "What Do We (Not) Know Theoretically about Solar Neutrino Fluxes?". \textit{Phys. Rev. Lett.}, 92, 121301.
    \bibitem{kamiokande1991}
    \textit{Kamiokande Collaboration}, 1991. "Real-time, directional measurement of ${}^{8} B$ solar neutrinos in the Kamiokande II detector". \textit{Phys. Rev.}, D44, 2241.
    \bibitem{superk2016}
    \textit{Super-Kamiokande Collaboration}, 2016. "Solar Neutrino Measurements in Super-Kamiokande-IV". \textit{Phys. Rev.}, D94, 052010.
    \bibitem{sage1991}
    \textit{SAGE Collaboration}, 1991. "Search for Neutrinos from the Sun Using the Reaction ${}^{71}$Ga($\nu_{e}$,$e^{-}$)${}^{71}$Ge". \textit{Phys. Rev. Lett.}, 67, 3332.
    \bibitem{gallex1999}
    \textit{GALLEX Collaboration}, 1999. "GALLEX solar neutrino observations: results for GALLEX IV". \textit{Phys. Lett.}, B447, 127.
    \bibitem{sno2001}
    \textit{SNO Collaboration}, 2001. "Measurement of the Rate of $\nu_{e} +d \rightarrow p+p+e^{-}$ Interactions Produced by ${}^{8}$B Solar Neutrinos at the Sudbury Neutrino Observatory". \textit{Phys. Rev. Lett.}, 87, 071301.
    \bibitem{hirata1998}
    Hirata K. S. \textit{et al.}, 1988. "Experimental Study of the Atmospheric Neutrino Flux". \textit{Phys. Lett.}, B205, 416.
    \bibitem{casper1991}
    Casper D. \textit{et al.}, 1991. "Measurement of Atmospheric Neutrino Composition with the IMB-3 Detector". \textit{Phys. Rev. Lett.}, 66, 2561.
    \bibitem{macro1998}
    \textit{MACRO Collaboration}, 1998. "Measurement of the Atmospheric Neutrino-induced Upgoing Muon Flux Using MACRO". \textit{Phys. Lett}, B434, 451.
    \bibitem{pontecorvo1957}
    Pontecorvo B., 1957. "Mesonium and Antimesonium". \textit{JETP}, 6, 429.
    \bibitem{gellmann1955}
    Gell-Mann M., Pais A., 1955. "Behavior of Neutral Particles under Charge Conjugation". \textit{Phys. Rev.}, 97, 1387.
    \bibitem{mns1962}
    Maki Z. Nakagawa M., Sakata S., 1962. "Remarks on the Unified Model of Elementary Particles". \textit{Prog. of Theo. Phys.}, 28, 870.
    \bibitem{pontecorvo1967}
    Pontecorvo B., 1967. "Neutrino Experiments and the Problem of Conservation of Leptonic". \textit{JETP}, 26, 984.
    \bibitem{pontecorvo1969}
    Gribov V., Pontecorvo B., 1969. "Neutrino Astronomy and Lepton Charge". \textit{Phys. Lett.}, 28B, 493.
    \bibitem{fulltheory70s}
    Bilenky S. M., Pontecorvo B., 1976. \textit{Phys. Lett.} B61, 248.\\
    Bilenky S. M., Pontecorvo B., 1976. \textit{Lett. Nuovo Cim.} 17, 569.\\ 
    Bilenky S. M., Pontecorvo B., 1978. \textit{Phys. Rep.} 41, 225.\\
    Eliezer S., Swift A., 1976. \textit{Nucl. Phys.} B105, 45.\\
    Fritzsch H., Minkowski P., 1976. \textit{Phys. Lett.} B62, 72.
    
\end{thebibliography}



\end{document}