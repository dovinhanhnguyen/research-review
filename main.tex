\documentclass[english]{article}
\usepackage[utf8]{inputenc}
\usepackage[T1]{fontenc}
\usepackage{babel}
\usepackage{amsmath}
\usepackage{graphicx}
\usepackage{fancyhdr}
\pagestyle{fancy}
\fancyhf{}
\renewcommand{\headrulewidth}{0pt}
\setlength{\headheight}{40pt} 

\begin{document}

\title{\bf(Title) 256 characters maximum, including spaces}

\author{(Authors) Taro NANKYOKU\textsuperscript{1}and Firstname LASTNAME\textsuperscript{2{*}}}

\maketitle
\thispagestyle{fancy}

\begin{center}
(Affiliations)\textsuperscript{1}National Institute of Polar Research, Research Organization of Information and Systems,\\
(Affiliations Address) 10–3, Midori-cho, Tachikawa, Tokyo 190-8518.\\
\textsuperscript{2}Department of Polar Science, School of Multidisciplinary Sciences, SOKENDAI (The Graduate University for Advanced Studies),\\
10–3, Midori-cho, Tachikawa, Tokyo 190-8518.\\
{*}Corresponding author(s). Taro Nankyoku (nankyokutaro@polar.ac.jp)

\end{center}

\vspace{1cm}

{\bf Abstract:} {\it500 words maximum (4000 character maximum).} The abstract should contain a concise summary and mention acquired and prepared data sets as well as possibilities for reusing those data sets. However, no new scientific findings should be presented here. (No citations may be included in the abstract.)


\begin{center}
{\bf 1. Background \& Summary}
\end{center}

 In this section, authors must explain the background of research that served as the basis for prepared data and explain the composition of that research with citations where necessary. They must also mention their motivation and purpose for preparing the data as well as the data’s value.{\it (1000 words maximum)}


\begin{center}
{\bf 2. Location (or Observation)}
\end{center}

 Using maps, authors must explain the locations at which they acquired data or made observations. For data or observations extending over a broad area, they must explain the geographical limits of the target area and the spatial resolution.

\begin{center}
{\bf 3. Methods}
\end{center}

 In this section, authors must describe the processing process they used when preparing data, such as the methods they used to acquire data or conduct observations, with citations as necessary.  In the case of observations, they must provide details on the equipment that were used and describe the hardware, software, and other tools that were used. They must also include photos of the equipment and illustrations of observation systems. If they used a specific type of software to prepare data, they must describe how that software was obtained.
 Authors must describe their methods in a manner that will make reproduction possible by a third party that acquires similar data and makes similar observations. Authors may cite other literature with regard to methods for data acquirement and observation; however, they must describe these methods as well as a method for reproducing the processing process. Authors may not omit the methods and processing process for the reason of having cited other literature.

\begin{center}
{\bf 4. Data Records}
\end{center}

 Authors should use this section to explain data records that pertain to this report.  They must explain the data files and their formats so that other researchers can reuse them by reading this section. Tables showing the data structure should be used for explanation of presented data sets.
 Additionally, even if prepared data formats use software that is generally applicable to all fields, authors must provide a method that is reusable by a third party by noting the manner in which the software was obtained, the version that was used, a method for reading the data using the software, and other relevant details.

\begin{center}
{\bf 5. Technical Validation}
\end{center}

 In this section, authors must provide the method they used to support the technical quality of the data sets. They should make their explanation using figures and tables when necessary. This is a required section; authors must provide information that legitimizes the reliability of their data.
One of the following items should be included:
\begin{itemize}
\item Details on the data gathering procedure or method used in quality evaluation
\item Statistical errors in the data
\end{itemize}

\begin{center}
{\bf 6. Usage Notes (optional)}
\end{center}

In this section, authors can describe any points that should be kept in mind when using data.

\begin{center}
{\bf 7. Competing interests (optional)}
\end{center}

 Data management plans designated by sources of funding may be provided here.(Because there are cases in which overseas funding organizations require the publication of data sets to be noted within one year, or the publication of those sets in this repository, as conditions for funding, those conditions can be described here.)

\begin{center}
{\bf 8. Figures (optional)}
\end{center}

 The figures and these legends can include up to a maximum of four that present main data points of the published data in a “quick look” format.

\begin{center}
{\bf 9. Tables (optional)}
\end{center}

 Tables may include main data points of the published data. However, this is acceptable only for tables arranged in a summarized format.

\begin{center}
{\bf Author contributions}
\end{center}

 This section must explain concisely the contributions made by each author in the following manner:
A did this and that.
B did this, that, and the other. 
The contribution of each author must be described.

\begin{center}
{\bf Acknowledgements}
\end{center}

 The acknowledgements must be kept concise and include contributions from people who are not counted among the authors. There is no need to include referees and editors in the acknowledgements.  Projects that gathered data and sources of funding can be included here.

\begin{center}
{\bf References}
\end{center}

 Provide bibliographic information for any works cited in the above sections, using the standard SIST referencing style.
 References are each numbered, ordered sequentially as they appear in the text.
 When cited in the text, reference numbers are superscript, not in brackets unless they are likely to be confused with a superscript number.
 All authors should be included in reference lists unless there are more than five, in which case only the first author should be given, followed by et al.

\begin{enumerate}
\item Author, A. \& Author, B. B. Article title. Proc. Natl Acad. Sci. USA. 2015, 101, 100–122. https://doi.org/10.15094/000XXXXX (\it Journal)
\item Author, A. A. et al. Book Title. Cambridge Univ. Press, 2004, 133 p. (\it Book)
\end{enumerate}

\begin{center}
{\bf Data Citations}
\end{center}

This section is for notation of bibliographic information that was provided in the data report. For this section, data repositories involving polar science will provide digital object identifiers (DOIs).
Bibliographic style for the data records described in the manuscript:

1. Lastname1, Initial1., Lastname2, Initial2., ...\& LastnameN, InitialN. Repository name. version, publisher, year, Dataset accession number or DOI.

\end{document}